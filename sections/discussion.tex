\cleardoublepage
\section{Discussion}\label{sec:discussion}

The following two sections qualitatively review the contribution with respect to the stated goals, and expose limitations of the current implementation as well as discuss flaws inherent to the design.

\subsection{Advantages}

minimal loc shows expressive power of language

pluggable transport

free indexing

immutability and db-as-value

\paragraph{Flexible data model.} Representing data as fully-normalized fact triples allows
In a traditional, row-oriented data model, modeling such data will lead to very wide, sparse tuples. Attribute-oriented data models can represent such data much more efficiently, and materialize only necessary attributes, at the cost of frequent multi-way joins across relations of potentially highly varying selectivities. \cite{gobel2019optimising}


\subsection{Limitations}

negation

Features such as combining multiple data sources in one query, simpler means to abstract and compose query fragments (called "rules" in Datomic), pull expressions,

Since the persistence of data is fundamental to any application, changes to the data layer.
Both cases have a dynamic that favours those building new systems as opposed to those that maintain or steward large existing systems.

-- determining if fact was superseded needs upfront specification of cardinality for field.
-- TODO does it make sense to default to cardinality one?
set vs multiset semantics -- leaning towards set on insertion, multiset on querying



framework vs library

composition of queries

temporal and general constraints

epochs/excision/selective deletion for compliance reasons
"The main transaction log contains only hashes and is immutable. All document content is stored in a dedicated document log that can be evicted by compaction." [crux]


persistence to disk

full unsynchronized read replication ("perception does not require coordination" [hickey])

split out transactor with possibility of primary-backup replication

problem: back dated changes. should we allow future dated changes? -- leaning towards no? or ignore? or cut off at "now" by default? we just defer answering this question as not a core concern of this poc


fully lazy datastructure allows transaprent access of past/superseded facts

real bitemporal query language (mix timelines in single query)

Second System Effect \cite{brooks1995mythical}

\paragraph{Open vs. closed system.} Incidental complexity within a closed system such as the one described in this work can be managed to stay at a minimal level because any outside components the system is interacting with is forced to adapt to the system's way. However, users of such a system eventually want to connect it to other things. Inevitably, doing so \emph{imports} some of the incidental complexities of those systems. It remains to be seen if we can build open systems without importing accidental complexity \cite{moffat16eve}.
