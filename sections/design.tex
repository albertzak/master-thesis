\section{Design}\label{sec:design}

The contribution of this work is divided into two main sections, design and implementation. The subsequent parts of this section first present various design problems of commonly employed data layer technologies. Deriving from their limitations, the next part paints a blissful picture of what an ideal data layer would look like (subsection~\ref{sec:goals}), while the following demarcates the scope of the contribution (subsection~\ref{sec:nongoals}). Finally, the conceptual model (subsection~\ref{sec:conceptual_model}) and the query language (subsection~\ref{sec:query_language}) are presented.

\subsection{Problems}

Traditional RDBMSs enforce a structural rigidity of fitting data into "tables", "rows", and "columns", entailing a lack of flexibility in dealing with sparse data, irregular data, hierarchies, or multi-valued attributes \cite{hickey2012dbvalue}. With the shift to data-intensive Single-Page Applications (SPAs), clients become limited peers to the database. Requirements of real time collaboration, evolution of the schema, and auditability of changes, together with limited request-response data loading mechanisms work together to increase the incidental complexity of application code.


\paragraph{\gls{SQL}.}
Despite the widespread use of SQL, it is not, in any version, an accurate reflection of the relational model \cite{codd1990relational} \cite{tarpit}.
Modeling domains in common \gls{RDBMS} requires distinguishing between entity and relationship \cite{chen1976entity}. A system in which choosing the structure for the data involves setting up "routes" between data instances (such as from a particular employee to a particular department) is access path dependent. A pure relational systems would need no problematic distinction between entity and relationship \cite{tarpit}.

SQL within the application's main programming language is usually treated as a second class citizen by resorting to concatenation of strings with a thin layer of sanitization. (Microsoft's LINQ \cite{meijer2006linq} is a notable exception.) Features such as materialized views are cumbersome to use because the language lacks powerful means of abstraction and composition \cite{sicp} which would enable recursively composing complex queries out of smaller queries.

\paragraph{Object-relational mismatch.}
\gls{ORM} of tuples from the database onto whichever data types are available in the backend programming languages, and increasingly also to the language in the front end in \gls{SPA}, as well as the choice of appropriate communication interfaces between at least three layers pose additional considerations. Duplicating structure in schema definitions and again in ORM code leads to a proliferation of types.

\paragraph{Time.}
Temporal databases are commonly optimized for analytics performance over large numeric datasets, e.g. sensor readings over time. Temporal extensions to SQL \cite{kulkarni2012temporal} are convoluted and do not address the fundamental problem at hand, which is that typical mutable-state systems obscure the sequence of events that brought the world into its current state.

\paragraph{Distribution.}
Thinking that data resides "over there" on a remote server breeds a "fear of round trips" and causes developers to accumulate incidental complexity in the form of performance optimizations related to caching \cite{hickey2012dbvalue}.

Yet, one cannot disavow the fundamental complexities of distributed systems: Network failures (latency, disconnection, offline context), consistency and availability choices taken by underlying technology which are not made explicit, and how different representations of data influence and determine what is and is not possible in terms of concurrent activity by different actors \cite{emerick2014api}.

\paragraph{Imperative fetching.}
Performing data loads via classic request-response semantics is not expressive enough because the act of fetching is made explicit and requires imperative calls to various endpoints. A developer of a \gls{SPA} with a \gls{REST} backend must decide along which boundaries to split the data endpoints. In traditional RESTful style, each \emph{resource} dictates its own endpoint with its own \gls{CRUD} actions. This approach places the burden of orchestrating a decomposed rich interaction on the clients \cite{calderwood15cqrs}, leading to the $n+1$ query problem when performing client-side joins, and an over-fetching of data which is not needed to display but happens to be provided by the endpoint.

At the other extreme, there exists one distinct endpoint providing the exact required data for each screen (notwithstanding that the concept of a \emph{screen} is vague in SPAs) which collects, filters, and joins the data on the server before serializing it as one tree, and sending it back to the client. This style causes incidental complexity through a proliferation of endpoints and ad-hoc schemata.

A middle-ground mixture of both styles is common, muddling the architectural waters and bringing both disadvantages together with even more code bloat. In any case, the act of fetching the data must be initiated by the client in an imperative way instead of data just being "here" in the client's view layer when needed.


\paragraph{Propagation of updates.}
The traditional understanding of queries for data at rest as in listing~\ref{lst:querying_data_at_rest} is that data exists at the database in one place, and queries are passed over there, usually in the form of SQL strings, and unroll themselves into a dataflow tree of operators, pulling data up the tree when they reach the leaves \cite{alvaro2015isee}.

\begin{lstlisting}[label={lst:querying_data_at_rest},morekeywords={employees,departments},caption=Querying data at rest \cite{alvaro2015isee}]
($\Pi$ [:name :department]
   ($\bowtie$ ($\sigma$ :name "Scott" employees)
      departments))
\end{lstlisting}

When the underlying data changes, queries do not get re-run automatically and clients are left looking at stale data. On the flip side, re-running the same query at different times is not guaranteed to yield the same results, as the database might have changed in the mean time. There are no facilities to temporally stabilize a basis for queries.

\subsection{Goals}

a flexible \emph{data layer} for small to medium size internal business applications for distributed collaboration in near-realtime.
out of the author's experience running a SaaS business designing, developing, and operating a
why no there are many special purpose dbs, mostly for performance.
but none optimized for "data just being there when you need it"

with least amount of code. but still backed by relational guarantees (acid)

needs different kind of db.
needs reinvention of the whole data layer.



To reduce some of the complexities in question, moseley and Marks recommend adopting functional and declarative programming constructs, along with a fully relational data model \cite{tarpit}.

\paragraph{Streaming relational queries.}
Recall the querying semantics for data at rest in \autoref{lst:querying_data_at_rest}. What is needed for rich interactions is a different model of streaming queries for data in motion: What is "just there" is not the data, but the query — which is \emph{instantiated in the network}, and the data flows through the query instead of the other way around. \autoref{lst:querying_data_in_motion} gives an example of a streaming query.

\begin{lstlisting}[label={lst:querying_data_in_motion},morekeywords={email-source,contacts-source},caption=Querying data in motion \cite{alvaro2015isee}]
($\Pi$ [:ip]
   ($\bowtie$ ($\sigma$ (classifier email :spam)
         email-source)
       contacts-source))
\end{lstlisting}




\paragraph{Notion of memory.}
with strict auditability (log+queryable past)
free auditing of everything, time travel
"save everything", query later
6NF EAV "Event Sourcing" Data Model w/ Explicit Time and Memory: "History of structured Facts"

\paragraph{Transactional guarantees.}
Transactions (ACID)


\paragraph{Adaptivity and dynamism.}
schemaless


\paragraph{Distribution.}
The handling of data between clients and servers should be as declarative as possible: Clients declare what data they are interested in, and the infrastrucutre takes care of fetching, caching, handling updates, unsubscribe

Isomorphic + Homoiconic: Single Language: ClojureScript on Lumo/Browser




`'\subsection{Non-goals}\label{sec:nongoals}


\paragraph{Efficiency.}
No attention is paid to the efficiency of compute and memory usage. Tradeoffs are almost always made in favor of clarity concerning the mapping between conceptual model and implementation of the proof of concept. The only major optimization is the fact that the triple index structure exists, leastwise it doubles as the simplest possible way to access arbitrary data without the overhead of parsing and executing a query.

Custom indexing strategies, e.g. ways to maintain a phonetic index to query for people's names, do not need to be part of the database design, because the triple indexing scheme is general enough to allow arbitrary access to the data in a manner that is efficient enough without having to declare indices upfront.

\paragraph{First class bitemporality.}
Another non-goal is the efficiency, primacy, and expressive power of the provided bitemporal affordances. A vast amount of previous work exists on relatively complex attempts to add efficient bitemporal semantics to relational databases \cite{snodgrass1996adding,jensen1999temporal,kulkarni2012temporal} notably for use in bitemporal constraints \cite{doucet1997using} or within complex queries, and of bitemporality as a first class concept in production rule systems \cite{aref2015design}. Bitemporality in the described system is only secondary. Access paths are optimized for the most recent view of the data, while bitemporality is meant to be used for infrequent auditing purposes. There is no bitemporal index, consequently issuing queries with temporal modifiers causes a sequential scan of the log.

\paragraph{Standard protocols.}
Lastly, the design explicitly avoids compliance with existing proliferated standards around the handling of data such as SQL, \gls{XML}, \gls{JSON}, \gls{REST}, etc. to instead allow quick exploration into different paradigms unburdened by past decisions.



\subsection{Conceptual model}\label{sec:conceptual_model}

The data model of the presented system is extremely simple. There is no requirement to design a schema or to differentiate between entities and relationships. Yet, \emph{all} structured data can be represented in the system as long as data is fully normalized (6NF).

\begin{itemize}
  \item The basic unit of information is a \emph{fact}, a triple \lisp{[e a v]} containing values representing \emph{entity}, \emph{attribute}, and \emph{value}.

  \item Over time, facts are \emph{asserted} and \emph{retracted}, accreted as part of a \emph{transaction}.

  \item The database and all changes to its state over time are fully described by the transaction log of assertions and retractions of facts.
\end{itemize}

\paragraph{Indexing.}

EAV systems commonly keep a number of sorted indices (see table \ref{tbl:indices}) to allow the data to be retrieved from multiple "angles" or directions, depending on the need of the query. Index structures are named after the \emph{nesting order} in which the elements of the facts are arranged. Not all database systems maintain the same indices. In this case, the system keeps four indices covering the following common use cases:
\begin{itemize}
  \item EAVT, the canonical order, which \emph{maps} an entity to its attributes like a document,
  \item AEVT, for finding entities which \emph{have} a certain attribute set
  \item AVET, for \emph{filtering} entities by a known attribute set to a known value,
  \item VAET, for \emph{searching} over all attributes of all entities by a known value.
\end{itemize}


\newcolumntype{s}{>{\hsize=.5\hsize}X}

\begin{table}[]
  \caption{Impact of the index sort order on the area of application}
  \begin{tabularx}{\textwidth}{|l|s|s|X|}
  \hline
  \textbf{index} & \textbf{name}               & \textbf{feels like}      & \textbf{good for}                                     \\ \hline
  EAVT          & "entity-oriented"           & document store           & accessing various attributes of a known entity        \\ \hline
  AEVT          & "attribute-entity-oriented" & column store             & accessing the same attribute of various entities      \\ \hline
  AVET          & "attribute-value-oriented"  & filtering a column store & finding entities by the value of a specific attribute \\ \hline
  VAET          & "value-oriented"            & searching everything     & searching over all values, regardless of attribute    \\ \hline
  \end{tabularx}
  \label{tbl:indices}
\end{table}

For example, here is a simple example to pull out the name of a known patient, using only the \lisp{get-in} function of the Clojure core library on the \lisp{:eavt} index:

\begin{center}
  \lisp{(get-in db [:eavt :patient/91 :name])}
\end{center}

One can also leave out the attribute, and get back a map (as a conceptual \emph{document}) containing all known attributes related to that entity.

\begin{center}
  \lisp{(get-in db [:eavt :patient/91])}
\end{center}

Performing a search by name over all patients is similarly trivial using the \lisp{:avet} index, with the result

\begin{center}
  \lisp{(get-in db [:avet :name "Hye-mi"])}
\end{center}


\cleardoublepage
\subsection{Query language}\label{sec:query_language}

The query language of the system is a greatly simplified language modeled after the pattern matching relational query language used in Datomic, which is in turn a Lisp variant of the Datalog \cite{abiteboul1988datalog} language expressed in of Clojure's \gls{edn}.

The choice of language is arbitrary -- any relational language would suffice -- and the core of the database does not depend on any query language capabilities Modeling the language after the one used in Datomic was chosen because because not only has the edn notation become a de-facto standard for other EAV databases like Crux, EVA, and Datascript, but because the shape of each query clause maps naturally to the representation of a fact in canonical EAV order.

See listing~\ref{lst:example_query} for an query consisting of four query clauses (the \lisp{:where} part) performing an implicit join, and a final projection (\lisp{:find}) to extract the values bound to the \emph{\gls{lvar}} symbols \lisp{?name} and \lisp{?location}. For example, the query clause \lisp{[?p :name ?name]} applied to the fact \lisp{[:person/123 :name "Hye-mi"]} would result in \emph{binding} the lvar \lisp{?p} to the value \lisp{:person/123}, and the lvar \lisp{?name} to the value \lisp{"Hye-mi"}. Other clauses are bound likewise. Note that multiple occurrences of the same lvar prompt \emph{unification} with the same value, creating an implicit \emph{join}. The order of the query clauses has no semantic meaning.

Performing a query entails applying the \lisp{q} function to a database value and a query. Clients can thus decide whether to leverage the query language via loading a library, or just access the data via the index structures directly.

\begin{lstlisting}[label={lst:example_query},caption="Who from Ulsan is working for whom?"]
'[:find [?name ?company]
  :where [[?p :works-for ?e]
          [?e :name ?company]
          [?p :name ?name]
          [?p :location "Ulsan"]]]
\end{lstlisting}

\paragraph{Temporal and bitemporal queries.}
As stated in section~\ref{sec:nongoals}, the (bi-)temporal aspects of the described system are secondary -- they are to be used for infrequent auditing purposes. Consequently, the design of the indexing and query mechanisms can be greatly simplified be forgoing bitemporal indexing strategies such as \cite{nascimento1995ivtt}.

As the query function simply takes a database as a \emph{value}, a \emph{filtering function} can be applied the the database beforehand. The \lisp{keep} function in listing~\ref{lst:queryfilter} returns a structurally shared and lazy copy of the database filtered by arbitrary bounds of the relevant timestamps $t_x$ and $t_v$.

\begin{lstlisting}[label={lst:queryfilter},caption=Applying a temporal filter before querying,morekeywords={keep,q,<,>}]
  (q (keep
       ($\lambda$ [$t_x$ $t_v$]
         (and (> $t_v$ 300) (< $t_v$ 500)
              (< $t_x$ 700)))
       db) query)
  \end{lstlisting}


\paragraph{Per-entity history.} A common use case in auditing is to retrieve the \emph{history} of all attributes related to a given entity over time. The \lisp{history} function takes a database value (optionally composed with a filtering function as described above) and an entity value, and returns an ordered slice of the log with transactions relevant to the requested entity. Note that it does not make sense to create a new database value from a history log, because that would just result in only the latest values being present in the index yet again.


\paragraph{Publication and subscription}

One of the goals states that clients should be able to declaratively subscribe to the \emph{live result set} of a query. The results and the query itself will change over the duration of a client's session. Each change triggers an immediate re-render of the UI. Conceptually, clients \emph{install} their \emph{subscription queries} on the server, and the infrastructure will re-run the subscription query whenever the underlying data changes and notify the client of the changed results. The design does not prescribe whether or not to replicate past (i.e. superseded or retracted) facts, thus greatly simplifying the proof-of-concept implementation by deferring concerns such as diffing, authorization, and the decision of what exactly to replicate to the clients to the developer customizing this data layer to their use case.

\paragraph{Security.} While extreme dynamism may be warranted in a high-trust environment, a real-world application may interact with some malicious entities and thus needs a means to restrict queries on the server side. In a real-world application, clients would need to authenticate themselves and the server would authorize publication based on access rules. Yet, there is no simple way to statically analyze queries submitted by the client for safety properties, but the server can control which facts are allowed to be replicated to a client. A publication might, for example, choose to not replicate facts with specific attributes, or transform facts to censor parts of the value.
