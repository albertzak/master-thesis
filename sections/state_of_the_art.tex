\cleardoublepage
\subsection{State of the art}\label{sec:sota}

This section gives and overview of the data layer paradigms

\paragraph{\gls{EAV}}
6nf / rdf / triplestores


\paragraph{Event sourcing.}
\gls{CQRS}
often in conjunction with \gls{DDD} \cite{evans2004domain}


\paragraph{Bitemporality}
While uni- or monotemporal databases allow querying along the transactional sequence of historic database states

support for bitemporality adds another separate axis

\gls{tv} \gls{tt}

When extending EAV databases with a first-class concept of time and/or transactions, such systems are sometimes referred to using the ambiguous abbreviation \emph{EAVT}, with the \emph{T} variously referring to either the addition of transaction time, valid time or other \emph{domain time} values (or a combination thereof) \cite{huser2013desiderata} or the \emph{T} may hint at the concept of first-class transactions, where a fact additionally carries the entity ID of its originating transaction to allow attaching arbitrary metadata to events.

\paragraph{Functional-reactive programming.}

database as a value \cite{hickey2012dbvalue}
live queries

Continuing on to span the \gls{FRP} paradigm from the server towards the client,
\cite{reynders2014multi}

When constructing the view layer out of pure functions The view as a pure function of state $v=f(s)$
Each component specifies what data it needs, composing components needn't know inner workings -- but bundled with graphql
