\cleardoublepage{}

\section*{Abstract}

Growing customer demands lead to increased incidental complexity of data-intensive distributed applications. Relational Database Management Systems (RDBMS), especially those implementing Structured Query Language (SQL) possess performant but destructive-by-default write semantics. Modeling domains such as healthcare in classic RDBMS leads to an explosion in the number of columns and tables. Their structure has to be known in advance, discouraging an explorative development process. Requirements of real time collaboration, auditability of changes, and evolution of the schema push the limit of established paradigms.

The core of the data layer presented in this thesis is a simple relational model based on facts in the form of Entity-Attribute-Value (EAV) triples. A central append-only immutable log accretes these facts via assertions and retractions. Transactions guarantee atomicity/consistency/isolation/durability (ACID) and are themselves first-class queryable entities carrying arbitrary meta facts, realizing strict bitemporal auditability of all changes by keeping two timestamps: transaction time $t_x$ and valid time $t_v$. Changes are replicated to clients which subscribe to the result of a query. Multiple incrementally maintained indices (EAVT, AEVT, AVET, VAET) grant efficient direct access to tuples, regarding the database analogous to a combined graph, column, and document store. The database itself is an immutable value which can be passed within the program and queried locally.

This work demonstrates the feasibility of implementing various desirable features in less than 400 lines of Clojure: bitemporality, audit logging, transactions, server-client reactivity, consistency criteria, derived facts, and a simple relational query language based on Datalog.
